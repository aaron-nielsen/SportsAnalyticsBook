\documentclass[]{article}
\usepackage{lmodern}
\usepackage{amssymb,amsmath}
\usepackage{ifxetex,ifluatex}
\usepackage{fixltx2e} % provides \textsubscript
\ifnum 0\ifxetex 1\fi\ifluatex 1\fi=0 % if pdftex
  \usepackage[T1]{fontenc}
  \usepackage[utf8]{inputenc}
\else % if luatex or xelatex
  \ifxetex
    \usepackage{mathspec}
  \else
    \usepackage{fontspec}
  \fi
  \defaultfontfeatures{Ligatures=TeX,Scale=MatchLowercase}
\fi
% use upquote if available, for straight quotes in verbatim environments
\IfFileExists{upquote.sty}{\usepackage{upquote}}{}
% use microtype if available
\IfFileExists{microtype.sty}{%
\usepackage{microtype}
\UseMicrotypeSet[protrusion]{basicmath} % disable protrusion for tt fonts
}{}
\usepackage[margin=1in]{geometry}
\usepackage{hyperref}
\hypersetup{unicode=true,
            pdfborder={0 0 0},
            breaklinks=true}
\urlstyle{same}  % don't use monospace font for urls
\usepackage{color}
\usepackage{fancyvrb}
\newcommand{\VerbBar}{|}
\newcommand{\VERB}{\Verb[commandchars=\\\{\}]}
\DefineVerbatimEnvironment{Highlighting}{Verbatim}{commandchars=\\\{\}}
% Add ',fontsize=\small' for more characters per line
\usepackage{framed}
\definecolor{shadecolor}{RGB}{248,248,248}
\newenvironment{Shaded}{\begin{snugshade}}{\end{snugshade}}
\newcommand{\AlertTok}[1]{\textcolor[rgb]{0.94,0.16,0.16}{#1}}
\newcommand{\AnnotationTok}[1]{\textcolor[rgb]{0.56,0.35,0.01}{\textbf{\textit{#1}}}}
\newcommand{\AttributeTok}[1]{\textcolor[rgb]{0.77,0.63,0.00}{#1}}
\newcommand{\BaseNTok}[1]{\textcolor[rgb]{0.00,0.00,0.81}{#1}}
\newcommand{\BuiltInTok}[1]{#1}
\newcommand{\CharTok}[1]{\textcolor[rgb]{0.31,0.60,0.02}{#1}}
\newcommand{\CommentTok}[1]{\textcolor[rgb]{0.56,0.35,0.01}{\textit{#1}}}
\newcommand{\CommentVarTok}[1]{\textcolor[rgb]{0.56,0.35,0.01}{\textbf{\textit{#1}}}}
\newcommand{\ConstantTok}[1]{\textcolor[rgb]{0.00,0.00,0.00}{#1}}
\newcommand{\ControlFlowTok}[1]{\textcolor[rgb]{0.13,0.29,0.53}{\textbf{#1}}}
\newcommand{\DataTypeTok}[1]{\textcolor[rgb]{0.13,0.29,0.53}{#1}}
\newcommand{\DecValTok}[1]{\textcolor[rgb]{0.00,0.00,0.81}{#1}}
\newcommand{\DocumentationTok}[1]{\textcolor[rgb]{0.56,0.35,0.01}{\textbf{\textit{#1}}}}
\newcommand{\ErrorTok}[1]{\textcolor[rgb]{0.64,0.00,0.00}{\textbf{#1}}}
\newcommand{\ExtensionTok}[1]{#1}
\newcommand{\FloatTok}[1]{\textcolor[rgb]{0.00,0.00,0.81}{#1}}
\newcommand{\FunctionTok}[1]{\textcolor[rgb]{0.00,0.00,0.00}{#1}}
\newcommand{\ImportTok}[1]{#1}
\newcommand{\InformationTok}[1]{\textcolor[rgb]{0.56,0.35,0.01}{\textbf{\textit{#1}}}}
\newcommand{\KeywordTok}[1]{\textcolor[rgb]{0.13,0.29,0.53}{\textbf{#1}}}
\newcommand{\NormalTok}[1]{#1}
\newcommand{\OperatorTok}[1]{\textcolor[rgb]{0.81,0.36,0.00}{\textbf{#1}}}
\newcommand{\OtherTok}[1]{\textcolor[rgb]{0.56,0.35,0.01}{#1}}
\newcommand{\PreprocessorTok}[1]{\textcolor[rgb]{0.56,0.35,0.01}{\textit{#1}}}
\newcommand{\RegionMarkerTok}[1]{#1}
\newcommand{\SpecialCharTok}[1]{\textcolor[rgb]{0.00,0.00,0.00}{#1}}
\newcommand{\SpecialStringTok}[1]{\textcolor[rgb]{0.31,0.60,0.02}{#1}}
\newcommand{\StringTok}[1]{\textcolor[rgb]{0.31,0.60,0.02}{#1}}
\newcommand{\VariableTok}[1]{\textcolor[rgb]{0.00,0.00,0.00}{#1}}
\newcommand{\VerbatimStringTok}[1]{\textcolor[rgb]{0.31,0.60,0.02}{#1}}
\newcommand{\WarningTok}[1]{\textcolor[rgb]{0.56,0.35,0.01}{\textbf{\textit{#1}}}}
\usepackage{graphicx,grffile}
\makeatletter
\def\maxwidth{\ifdim\Gin@nat@width>\linewidth\linewidth\else\Gin@nat@width\fi}
\def\maxheight{\ifdim\Gin@nat@height>\textheight\textheight\else\Gin@nat@height\fi}
\makeatother
% Scale images if necessary, so that they will not overflow the page
% margins by default, and it is still possible to overwrite the defaults
% using explicit options in \includegraphics[width, height, ...]{}
\setkeys{Gin}{width=\maxwidth,height=\maxheight,keepaspectratio}
\IfFileExists{parskip.sty}{%
\usepackage{parskip}
}{% else
\setlength{\parindent}{0pt}
\setlength{\parskip}{6pt plus 2pt minus 1pt}
}
\setlength{\emergencystretch}{3em}  % prevent overfull lines
\providecommand{\tightlist}{%
  \setlength{\itemsep}{0pt}\setlength{\parskip}{0pt}}
\setcounter{secnumdepth}{0}
% Redefines (sub)paragraphs to behave more like sections
\ifx\paragraph\undefined\else
\let\oldparagraph\paragraph
\renewcommand{\paragraph}[1]{\oldparagraph{#1}\mbox{}}
\fi
\ifx\subparagraph\undefined\else
\let\oldsubparagraph\subparagraph
\renewcommand{\subparagraph}[1]{\oldsubparagraph{#1}\mbox{}}
\fi

%%% Use protect on footnotes to avoid problems with footnotes in titles
\let\rmarkdownfootnote\footnote%
\def\footnote{\protect\rmarkdownfootnote}

%%% Change title format to be more compact
\usepackage{titling}

% Create subtitle command for use in maketitle
\providecommand{\subtitle}[1]{
  \posttitle{
    \begin{center}\large#1\end{center}
    }
}

\setlength{\droptitle}{-2em}

  \title{}
    \pretitle{\vspace{\droptitle}}
  \posttitle{}
    \author{}
    \preauthor{}\postauthor{}
    \date{}
    \predate{}\postdate{}
  
\usepackage{graphicx}				% Use pdf, png, jpg, or eps§ with pdflatex; use eps in DVI mode
\usepackage{amssymb,amsmath}
\usepackage{natbib}
\usepackage{bm}
\usepackage{xcolor}
\usepackage{tikz}
\usepackage{ctable}
% \usepackage{newtxtext} % Times-like font
% \usepackage{titlesec}
% \titleformat*{\section}{\Large\bfseries\sffamily}
% \titleformat*{\subsection}{\large\bfseries\sffamily}

%%%%%%%%%%%
% Math Macros
\newcommand{\bmA}{\ensuremath{\bm A}}
\newcommand{\bma}{\ensuremath{\bm a}}
\newcommand{\bmB}{\ensuremath{\bm B}}
\newcommand{\bmb}{\ensuremath{\bm b}}
\newcommand{\bmC}{\ensuremath{\bm C}}
\newcommand{\bmc}{\ensuremath{\bm c}}
\newcommand{\bmD}{\ensuremath{\bm D}}
\newcommand{\bmd}{\ensuremath{\bm d}}
\newcommand{\bmE}{\ensuremath{\bm E}}
\newcommand{\bme}{\ensuremath{\bm e}}
\newcommand{\bmF}{\ensuremath{\bm F}}
\newcommand{\bmG}{\ensuremath{\bm G}}
\newcommand{\bmg}{\ensuremath{\bm g}}
\newcommand{\bmH}{\ensuremath{\bm H}}
\newcommand{\bmI}{\ensuremath{\bm I}}
\newcommand{\bmJ}{\ensuremath{\bm J}}
\newcommand{\bmK}{\ensuremath{\bm K}}
\newcommand{\bmL}{\ensuremath{\bm L}}
\newcommand{\bmM}{\ensuremath{\bm M}}
\newcommand{\bmP}{\ensuremath{\bm P}}
\newcommand{\bmQ}{\ensuremath{\bm Q}}
\newcommand{\bmq}{\ensuremath{\bm q}}
\newcommand{\bmR}{\ensuremath{\bm R}}
\newcommand{\bmr}{\ensuremath{\bm r}}
\newcommand{\bmS}{\ensuremath{\bm S}}
\newcommand{\bms}{\ensuremath{\bm s}}
\newcommand{\bmT}{\ensuremath{\bm T}}
\newcommand{\bmt}{\ensuremath{\bm t}}
\newcommand{\bmU}{\ensuremath{\bm U}}
\newcommand{\bmu}{\ensuremath{\bm u}}
\newcommand{\bmV}{\ensuremath{\bm V}}
\newcommand{\bmv}{\ensuremath{\bm v}}
\newcommand{\bmW}{\ensuremath{\bm W}}
\newcommand{\bmw}{\ensuremath{\bm w}}
\newcommand{\bmX}{\ensuremath{\bm X}}
\newcommand{\bmx}{\ensuremath{\bm x}}
\newcommand{\bmY}{\ensuremath{\bm Y}}
\newcommand{\bmy}{\ensuremath{\bm y}}
\newcommand{\bmZ}{\ensuremath{\bm Z}}
\newcommand{\bmz}{\ensuremath{\bm z}}


\newcommand{\bmalpha}{\ensuremath{\bm{\alpha}}}
\newcommand{\bmbeta}{\ensuremath{\bm{\beta}}}
\newcommand{\bmdelta}{\ensuremath{\bm{\delta}}}
\newcommand{\bmeta}{\ensuremath{\bm{\eta}}}
\newcommand{\bmepsilon}{\ensuremath{\bm{\epsilon}}}
\newcommand{\bmGamma}{\ensuremath{\bm{\Gamma}}}
\newcommand{\bmgamma}{\ensuremath{\bm{\gamma}}}
\newcommand{\bmLambda}{\ensuremath{\bm{\Lambda}}}
\newcommand{\bmmu}{\ensuremath{\bm{\mu}}}
\newcommand{\bmphi}{\ensuremath{\bm{\phi}}}
\newcommand{\bmSigma}{\ensuremath{\bm{\Sigma}}}
\newcommand{\bmtheta}{\ensuremath{\bm{\theta}}}
\newcommand{\bmzeta}{\ensuremath{\bm{\zeta}}}

\newcommand{\rank}{\ensuremath{\mathsf{rank}}}
\newcommand{\nullity}{\ensuremath{\mathsf{nullity}}}
\newcommand{\trace}{\ensuremath{\mathsf{tr}}}
\newcommand{\diag}{\ensuremath{\mathsf{diag}}}
\newcommand{\vecspan}{\ensuremath{\mathsf{span}}}

\newcommand{\mT}{\ensuremath{\mathsf{T}}}

\newcommand{\bbh}{\ensuremath{\hat{\bmbeta}}}
\newcommand{\Rn}{\ensuremath{\mathbb{R}^n}}
\newcommand{\Rnp}{\ensuremath{\mathbb{R}^{n\times p}}}
\newcommand{\Pa}{\ensuremath{{\bmP}_{\bmA}}}
\newcommand{\Pb}{\ensuremath{{\bmP}_{\bmB}}}
\newcommand{\Pv}{\ensuremath{{\bmP}_\mathcal{V}}}
\newcommand{\Pw}{\ensuremath{{\bmP}_\mathcal{W}}}
\newcommand{\Px}{\ensuremath{{\bmP}_{\bmX}}}
\newcommand{\XtX}{\ensuremath{\bmX^\mT\bmX}}
\newcommand{\XtXinv}{\ensuremath{(\bmX^\mT\bmX)^{-1}}}
\newcommand{\gtb}{\ensuremath{\bmg^\mT\bmbeta}}

\DeclareMathOperator*{\argmin}{arg\,min}
\newcommand{\bbmx}{\ensuremath{\begin{bmatrix}}}
\newcommand{\ebmx}{\ensuremath{\end{bmatrix}}}

\newcommand{\E}{\ensuremath{\mathrm{E}}}
\newcommand{\Var}{\ensuremath{\mathrm{Var}}}
\newcommand{\Cov}{\ensuremath{\mathrm{Cov}}}

\newcommand{\FWER}{\ensuremath{\mathrm{FWER}}}
\newcommand{\PCER}{\ensuremath{\mathrm{PCER}}}
\newcommand{\FDR}{\ensuremath{\mathrm{FDR}}}
\newcommand{\sFWER}{\ensuremath{\mathrm{sFWER}}}
\usepackage{booktabs}
\usepackage{longtable}
\usepackage{array}
\usepackage{multirow}
\usepackage{wrapfig}
\usepackage{float}
\usepackage{colortbl}
\usepackage{pdflscape}
\usepackage{tabu}
\usepackage{threeparttable}
\usepackage{threeparttablex}
\usepackage[normalem]{ulem}
\usepackage{makecell}
\usepackage{xcolor}

\begin{document}

\hypertarget{isaacs-stuff}{%
\section{Isaac's stuff}\label{isaacs-stuff}}

\begin{center}\rule{0.5\linewidth}{\linethickness}\end{center}

\begin{center}\rule{0.5\linewidth}{\linethickness}\end{center}

\hypertarget{scraping}{%
\subsubsection{Scraping}\label{scraping}}

\begin{Shaded}
\begin{Highlighting}[]
\KeywordTok{library}\NormalTok{(dplyr)}
\end{Highlighting}
\end{Shaded}

\begin{verbatim}
## Warning: package 'dplyr' was built under R version 3.6.2
\end{verbatim}

\begin{verbatim}
## 
## Attaching package: 'dplyr'
\end{verbatim}

\begin{verbatim}
## The following objects are masked from 'package:stats':
## 
##     filter, lag
\end{verbatim}

\begin{verbatim}
## The following objects are masked from 'package:base':
## 
##     intersect, setdiff, setequal, union
\end{verbatim}

\begin{Shaded}
\begin{Highlighting}[]
\KeywordTok{library}\NormalTok{(rvest)}
\KeywordTok{library}\NormalTok{(tidyverse)}
\end{Highlighting}
\end{Shaded}

\begin{verbatim}
## Warning: package 'tidyverse' was built under R version 3.6.2
\end{verbatim}

\begin{verbatim}
## -- Attaching packages --------------------------------------- tidyverse 1.3.1 --
\end{verbatim}

\begin{verbatim}
## v ggplot2 3.3.5     v purrr   0.3.4
## v tibble  3.1.0     v stringr 1.4.0
## v tidyr   1.1.3     v forcats 0.5.1
## v readr   1.4.0
\end{verbatim}

\begin{verbatim}
## Warning: package 'ggplot2' was built under R version 3.6.2
\end{verbatim}

\begin{verbatim}
## Warning: package 'tibble' was built under R version 3.6.2
\end{verbatim}

\begin{verbatim}
## Warning: package 'tidyr' was built under R version 3.6.2
\end{verbatim}

\begin{verbatim}
## Warning: package 'readr' was built under R version 3.6.2
\end{verbatim}

\begin{verbatim}
## Warning: package 'purrr' was built under R version 3.6.2
\end{verbatim}

\begin{verbatim}
## Warning: package 'forcats' was built under R version 3.6.2
\end{verbatim}

\begin{verbatim}
## -- Conflicts ------------------------------------------ tidyverse_conflicts() --
## x dplyr::filter()         masks stats::filter()
## x readr::guess_encoding() masks rvest::guess_encoding()
## x dplyr::lag()            masks stats::lag()
\end{verbatim}

\begin{Shaded}
\begin{Highlighting}[]
\KeywordTok{library}\NormalTok{(kableExtra)}
\end{Highlighting}
\end{Shaded}

\begin{verbatim}
## Warning: package 'kableExtra' was built under R version 3.6.2
\end{verbatim}

\begin{verbatim}
## 
## Attaching package: 'kableExtra'
\end{verbatim}

\begin{verbatim}
## The following object is masked from 'package:dplyr':
## 
##     group_rows
\end{verbatim}

\hypertarget{wnba-scraping}{%
\subsection{wnba scraping}\label{wnba-scraping}}

\begin{Shaded}
\begin{Highlighting}[]
\NormalTok{wilson <-}\StringTok{ 'https://www.basketball-reference.com/wnba/players/w/wilsoa01w/gamelog/2022/'}
\NormalTok{wil_doc <-}\StringTok{ }\NormalTok{rvest}\OperatorTok{::}\KeywordTok{read_html}\NormalTok{(wilson)}

\NormalTok{wil_doc }\OperatorTok
\StringTok{  }\NormalTok{rvest}\OperatorTok{::}\KeywordTok{html_elements}\NormalTok{(., }\DataTypeTok{xpath =} \StringTok{"//*[(@id = 'div_wnba_pgl_basic')]"}\NormalTok{) }\OperatorTok
\StringTok{  }\NormalTok{rvest}\OperatorTok{::}\KeywordTok{html_table}\NormalTok{() ->}\StringTok{ }\NormalTok{wil}
\NormalTok{wil <-}\StringTok{ }\NormalTok{wil[[}\DecValTok{1}\NormalTok{]]}
\KeywordTok{head}\NormalTok{(wil)}
\end{Highlighting}
\end{Shaded}

\begin{verbatim}
## # A tibble: 6 x 28
##   Rk    Date   Age   Tm    ``    Opp   ``    GS    MP    FG    FGA   `FG%` `3P` 
##   <chr> <chr>  <chr> <chr> <chr> <chr> <chr> <chr> <chr> <chr> <chr> <chr> <chr>
## 1 1     2022-~ 25-2~ LVA   "@"   PHO   W (+~ 1     28:35 5     8     .625  0    
## 2 2     2022-~ 25-2~ LVA   ""    SEA   W (+~ 1     35:06 8     14    .571  1    
## 3 3     2022-~ 25-2~ LVA   "@"   WAS   L (-~ 1     29:56 4     11    .364  0    
## 4 4     2022-~ 25-2~ LVA   "@"   ATL   W (+~ 1     29:08 6     11    .545  0    
## 5 5     2022-~ 25-2~ LVA   ""    PHO   W (+~ 1     33:45 4     8     .500  0    
## 6 6     2022-~ 25-2~ LVA   ""    MIN   W (+~ 1     31:16 5     9     .556  1    
## # ... with 15 more variables: 3PA <chr>, 3P% <chr>, FT <chr>, FTA <chr>,
## #   FT% <chr>, ORB <chr>, DRB <chr>, TRB <chr>, AST <chr>, STL <chr>,
## #   BLK <chr>, TOV <chr>, PF <chr>, PTS <chr>, GmSc <chr>
\end{verbatim}

\begin{Shaded}
\begin{Highlighting}[]
\CommentTok{#wil2 <- mutate_all(wil, function(x) as.numeric(as.character(x)))}
\CommentTok{#mean(wil2['PTS'])}

\CommentTok{#wil$eFG<- (wil['FG'] + (0.5*wil['3P']))/wil['FGA']}
\CommentTok{#wil$eFG ![Screenshot]('~/Google Drive/My Drive/Sports Analytics/SportsAnalyticsBook/images/scraping1')}
\end{Highlighting}
\end{Shaded}

\begin{center}\rule{0.5\linewidth}{\linethickness}\end{center}

\hypertarget{edaprobability}{%
\subsubsection{EDA/Probability}\label{edaprobability}}

\begin{center}\rule{0.5\linewidth}{\linethickness}\end{center}

\hypertarget{baseball}{%
\subsubsection{Baseball}\label{baseball}}

\hypertarget{war-comparison-prob}{%
\paragraph{WAR comparison (Prob)}\label{war-comparison-prob}}

Link to WAR explaination:
\url{https://www.mlb.com/glossary/advanced-stats/wins-above-replacement}

Player X has a projected mean WAR of 3 with standard deviation of 2 and
player Y has a projected mean WAR of 1.5 with a standard deviation of 3.
Assume projected WAR is normally distributed. Q: What is the probability
that Player X outperforms Player Y? A: We want Pr(X\textgreater{}Y) or
Pr(X-Y\textgreater{}0).\\
Let Z = X-Y.\\
E{[}Z{]}=1.5 Var(Z)=5 Pr(Z\textgreater{}0)=1-Pr(Z \(\leq\) 0)

\begin{Shaded}
\begin{Highlighting}[]
\CommentTok{#Calculate probability Z<=0}
\NormalTok{pr <-}\StringTok{ }\KeywordTok{pnorm}\NormalTok{(}\DecValTok{0}\NormalTok{,}\FloatTok{1.5}\NormalTok{,}\KeywordTok{sqrt}\NormalTok{(}\DecValTok{5}\NormalTok{))}
\KeywordTok{print}\NormalTok{(}\DecValTok{1}\OperatorTok{-}\NormalTok{pr)}
\end{Highlighting}
\end{Shaded}

\begin{verbatim}
## [1] 0.7488325
\end{verbatim}

The Probability that Player X outperforms Player Y is 0.7488.

\hypertarget{injured-baserunner-prob}{%
\paragraph{Injured Baserunner (Prob)}\label{injured-baserunner-prob}}

A runner on first base with 2 out and nobody else on base will attempt
to steal second base on the first pitch 70\% of the time if he is fully
healthy but only 10\% of the time if he is playing through an injury.
Assume that 80\% of the player population is healthy. You see a randomly
selected runner not attempt a steal in this situation. Q: What is the
probability that the runner is playing through an injury? A: From Bayes
Theorem:

Pr(Injury given No Steal) = Pr(No Steal given Injury)*Pr(Injury)/P(No
Steal).

Pr(No Steal given Injury) = 1 - Pr(Steal given Injury) = 0.9.

Pr(Injury) = 1- Pr(Healthy) = 0.2.

Pr(No Steal) = Pr(No Steal given Injury)*Pr(Injury)+Pr(No Steal given
Healthy)*Pr(Healthy).

Pr(No Steal) = 0.9*0.2+0.7*0.8 = 0.74.

Therefore Pr(Injury given No Steal) = 0.9*0.2/0.74 = 0.243.

\hypertarget{ops-eda}{%
\paragraph{OPS (EDA)}\label{ops-eda}}

Q: Using the dataset, plot the leagues average OPS from every year in
the data to see the progression. A:

\begin{Shaded}
\begin{Highlighting}[]
\NormalTok{mlb =}\StringTok{ }\KeywordTok{read.csv}\NormalTok{(}\StringTok{'~/Google Drive/My Drive/Sports Analytics/SportsAnalyticsBook/data/mlb_team_stats_history.csv'}\NormalTok{)}
\KeywordTok{head}\NormalTok{(mlb)}
\end{Highlighting}
\end{Shaded}

\begin{verbatim}
##   yearID lgID teamID franchID divID Rank   G Ghome  W  L DivWin WCWin LgWin
## 1   1976   NL    ATL      ATL     W    6 162    81 70 92      N           N
## 2   1976   AL    BAL      BAL     E    2 162    81 88 74      N           N
## 3   1976   AL    BOS      BOS     E    3 162    81 83 79      N           N
## 4   1976   AL    CAL      ANA     W    4 162    81 76 86      N           N
## 5   1976   AL    CHA      CHW     W    6 161    80 64 97      N           N
## 6   1976   NL    CHN      CHC     E    4 162    81 75 87      N           N
##   WSWin   R   AB    H  X1B X2B X3B  HR  BB  SO  SB CS HBP SF  RA    BA  ER  ERA
## 1     N 620 5345 1309 1027 170  30  82 589 811  74 61  19 47 700 0.245 617 3.86
## 2     N 619 5457 1326  966 213  28 119 519 883 150 61  23 35 598 0.243 541 3.32
## 3     N 716 5511 1448 1004 257  53 134 500 832  95 70  29 59 660 0.263 571 3.52
## 4     N 550 5385 1265  969 210  23  63 534 812 126 80  42 48 631 0.235 551 3.36
## 5     N 586 5532 1410 1082 209  46  73 471 739 120 53  34 55 745 0.255 684 4.25
## 6     N 611 5519 1386 1041 216  24 105 490 834  74 74  30 41 728 0.251 643 3.93
##   CG SHO SV IPouts   HA HRA BBA SOA   E  DP    FP              name
## 1 33  13 27   4314 1435  86 564 818 167 151 0.973    Atlanta Braves
## 2 59  16 23   4406 1396  80 489 678 118 157 0.982 Baltimore Orioles
## 3 49  13 27   4374 1495 109 409 673 141 148 0.978    Boston Red Sox
## 4 64  15 17   4432 1323  95 553 992 150 139 0.977 California Angels
## 5 54  10 22   4344 1460  87 600 802 130 155 0.979 Chicago White Sox
## 6 27  12 33   4414 1511 123 490 850 140 145 0.978      Chicago Cubs
##                            park attendance BPF PPF teamIDBR teamIDlahman45
## 1 Atlanta-Fulton County Stadium     818179 106 108      ATL            ATL
## 2              Memorial Stadium    1058609  94  93      BAL            BAL
## 3                Fenway Park II    1895846 113 112      BOS            BOS
## 4               Anaheim Stadium    1006774  93  94      CAL            CAL
## 5                 Comiskey Park     914945 101 102      CHW            CHA
## 6                 Wrigley Field    1026217 108 109      CHC            CHN
##   teamIDretro
## 1         ATL
## 2         BAL
## 3         BOS
## 4         CAL
## 5         CHA
## 6         CHN
\end{verbatim}

\begin{Shaded}
\begin{Highlighting}[]
\CommentTok{# make new variables}
\NormalTok{mlb=}\KeywordTok{mutate}\NormalTok{(mlb,}\DataTypeTok{SLG=}\NormalTok{(X1B}\OperatorTok{+}\DecValTok{2}\OperatorTok{*}\NormalTok{X2B}\OperatorTok{+}\DecValTok{3}\OperatorTok{*}\NormalTok{X3B}\OperatorTok{+}\DecValTok{4}\OperatorTok{*}\NormalTok{HR)}\OperatorTok{/}\NormalTok{(AB))}
\NormalTok{mlb=}\KeywordTok{mutate}\NormalTok{(mlb,}\DataTypeTok{OBP=}\NormalTok{(H}\OperatorTok{+}\NormalTok{BB}\OperatorTok{+}\NormalTok{HBP)}\OperatorTok{/}\NormalTok{(AB}\OperatorTok{+}\NormalTok{BB}\OperatorTok{+}\NormalTok{HBP}\OperatorTok{+}\NormalTok{SF))}
\NormalTok{mlb=}\KeywordTok{mutate}\NormalTok{(mlb,}\DataTypeTok{OPS=}\NormalTok{OBP}\OperatorTok{+}\NormalTok{SLG)}

\CommentTok{# get avg ops}
\KeywordTok{summarize}\NormalTok{(mlb, }\DataTypeTok{Average =} \KeywordTok{mean}\NormalTok{(OPS,}\DataTypeTok{na.rm=}\NormalTok{T))}
\end{Highlighting}
\end{Shaded}

\begin{verbatim}
##     Average
## 1 0.7330384
\end{verbatim}

\begin{Shaded}
\begin{Highlighting}[]
\CommentTok{# get avg ops by year}
\KeywordTok{group_by}\NormalTok{(mlb, yearID)}\OperatorTok
\KeywordTok{summarize}\NormalTok{(}\DataTypeTok{Average =} \KeywordTok{mean}\NormalTok{(OPS, }\DataTypeTok{na.rm=}\NormalTok{T))}
\end{Highlighting}
\end{Shaded}

\begin{verbatim}
## # A tibble: 43 x 2
##    yearID Average
##     <int>   <dbl>
##  1   1976   0.681
##  2   1977   0.730
##  3   1978   0.702
##  4   1979   0.727
##  5   1980   0.714
##  6   1981   0.688
##  7   1982   0.712
##  8   1983   0.714
##  9   1984   0.707
## 10   1985   0.714
## # ... with 33 more rows
\end{verbatim}

\begin{Shaded}
\begin{Highlighting}[]
\KeywordTok{group_by}\NormalTok{(mlb, yearID)}\OperatorTok
\KeywordTok{summarize}\NormalTok{(}\DataTypeTok{Average =} \KeywordTok{mean}\NormalTok{(OPS, }\DataTypeTok{na.rm=}\NormalTok{T))}\OperatorTok\NormalTok{View}

\CommentTok{#create new dataset}
\NormalTok{mlbYr=}\KeywordTok{group_by}\NormalTok{(mlb, yearID)}\OperatorTok
\KeywordTok{summarize}\NormalTok{(}\DataTypeTok{Average =} \KeywordTok{mean}\NormalTok{(OPS, }\DataTypeTok{na.rm=}\NormalTok{T))}

\CommentTok{#plot it}
\KeywordTok{ggplot}\NormalTok{(mlbYr, }\KeywordTok{aes}\NormalTok{(}\DataTypeTok{x=}\NormalTok{yearID, }\DataTypeTok{y=}\NormalTok{ Average))}\OperatorTok{+}\KeywordTok{geom_point}\NormalTok{()}
\end{Highlighting}
\end{Shaded}

\includegraphics{22-isaac_files/figure-latex/unnamed-chunk-4-1.pdf}
Followup Q: What would cause the data to peak around the year 2000? A:
PED's

\hypertarget{run-variance-probability}{%
\paragraph{Run Variance (Probability)}\label{run-variance-probability}}

\begin{center}
\begin{tabular}{ c c c c c }
 Runs Scored & Probability \\ 
 0 & 0.55 \\  
 1 & 0.25 \\
 2 & 0.15 \\
 3 & 0.05
\end{tabular}
\end{center}

\begin{Shaded}
\begin{Highlighting}[]
\NormalTok{col1 =}\StringTok{ }\KeywordTok{c}\NormalTok{(}\StringTok{'Runs Scored'}\NormalTok{,}\StringTok{'Probability'}\NormalTok{)}
\NormalTok{col2 =}\StringTok{ }\KeywordTok{c}\NormalTok{(}\StringTok{'0'}\NormalTok{,}\FloatTok{0.55}\NormalTok{)}
\NormalTok{col3 =}\StringTok{ }\KeywordTok{c}\NormalTok{(}\StringTok{'1'}\NormalTok{,}\FloatTok{0.25}\NormalTok{)}
\NormalTok{col4=}\KeywordTok{c}\NormalTok{(}\StringTok{'2'}\NormalTok{,}\FloatTok{0.15}\NormalTok{)}
\NormalTok{col5=}\KeywordTok{c}\NormalTok{(}\StringTok{'3'}\NormalTok{,}\FloatTok{0.05}\NormalTok{)}
\NormalTok{runs <-}\StringTok{ }\KeywordTok{data.frame}\NormalTok{(col1,col2,col3,col4,col5)}
\KeywordTok{colnames}\NormalTok{(runs) <-}\StringTok{ }\OtherTok{NULL}
\NormalTok{runs}
\end{Highlighting}
\end{Shaded}

\begin{verbatim}
##                                  
## 1 Runs Scored    0    1    2    3
## 2 Probability 0.55 0.25 0.15 0.05
\end{verbatim}

\begin{Shaded}
\begin{Highlighting}[]
\KeywordTok{kbl}\NormalTok{(runs)}
\end{Highlighting}
\end{Shaded}

\begin{tabular}[t]{l|l|l|l|l}
\hline
Runs Scored & 0 & 1 & 2 & 3\\
\hline
Probability & 0.55 & 0.25 & 0.15 & 0.05\\
\hline
\end{tabular}

\begin{center}\rule{0.5\linewidth}{\linethickness}\end{center}

\hypertarget{tennis}{%
\subsubsection{Tennis}\label{tennis}}

Link for brief explanation of tennis scoring:
\url{https://www.sportingnews.com/us/tennis/news/tennis-scoring-explained-rules-system-points-terms/7uzp2evdhbd11obdd59p3p1cx}

\hypertarget{probability-of-winning-a-game-prob}{%
\paragraph{Probability of Winning a Game
(Prob)}\label{probability-of-winning-a-game-prob}}

The formula for the probability of a tennis player winning a game (from
Analyzing Wimbledon) is given by
\(\frac{p^4*(-8*p^3+28*p^2-34*p+15)}{p^2+(1-p)^2}\) where \(p\) is the
probability of a player winning their service point. Q: If a player wins
their service points 62\% of the time, what is the probability they win
the game? A:

\begin{Shaded}
\begin{Highlighting}[]
\NormalTok{p <-}\StringTok{ }\FloatTok{0.62}
\NormalTok{pr_game <-}\StringTok{ }\NormalTok{(p}\OperatorTok{^}\DecValTok{4}\OperatorTok{*}\NormalTok{(}\OperatorTok{-}\DecValTok{8}\OperatorTok{*}\NormalTok{p}\OperatorTok{^}\DecValTok{3}\OperatorTok{+}\DecValTok{28}\OperatorTok{*}\NormalTok{p}\OperatorTok{^}\DecValTok{2-34}\OperatorTok{*}\NormalTok{p}\OperatorTok{+}\DecValTok{15}\NormalTok{))}\OperatorTok{/}\NormalTok{(p}\OperatorTok{^}\DecValTok{2}\OperatorTok{+}\NormalTok{(}\DecValTok{1}\OperatorTok{-}\NormalTok{p)}\OperatorTok{^}\DecValTok{2}\NormalTok{)}
\NormalTok{pr_game}
\end{Highlighting}
\end{Shaded}

\begin{verbatim}
## [1] 0.7758627
\end{verbatim}

\hypertarget{graph-example-of-probability-of-winning-point-vs-probability-of-winning-game-prob}{%
\paragraph{Graph Example of Probability of Winning Point vs Probability
of Winning Game
(Prob)}\label{graph-example-of-probability-of-winning-point-vs-probability-of-winning-game-prob}}

\begin{Shaded}
\begin{Highlighting}[]
\NormalTok{game <-}\StringTok{ }\KeywordTok{c}\NormalTok{(}\DecValTok{0}\NormalTok{)}
\NormalTok{pr <-}\StringTok{ }\DecValTok{1}\OperatorTok{:}\DecValTok{100}
\ControlFlowTok{for}\NormalTok{(x }\ControlFlowTok{in}\NormalTok{ pr) \{}
\NormalTok{  p <-}\StringTok{ }\NormalTok{pr}\OperatorTok{/}\DecValTok{100}
\NormalTok{  pr_game <-}\StringTok{ }\NormalTok{(p}\OperatorTok{^}\DecValTok{4}\OperatorTok{*}\NormalTok{(}\OperatorTok{-}\DecValTok{8}\OperatorTok{*}\NormalTok{p}\OperatorTok{^}\DecValTok{3}\OperatorTok{+}\DecValTok{28}\OperatorTok{*}\NormalTok{p}\OperatorTok{^}\DecValTok{2-34}\OperatorTok{*}\NormalTok{p}\OperatorTok{+}\DecValTok{15}\NormalTok{))}\OperatorTok{/}\NormalTok{(p}\OperatorTok{^}\DecValTok{2}\OperatorTok{+}\NormalTok{(}\DecValTok{1}\OperatorTok{-}\NormalTok{p)}\OperatorTok{^}\DecValTok{2}\NormalTok{)}
\NormalTok{  game <-}\StringTok{ }\KeywordTok{c}\NormalTok{(game,pr_game)}
\NormalTok{\}}
\NormalTok{game[}\DecValTok{1}\NormalTok{]}
\end{Highlighting}
\end{Shaded}

\begin{verbatim}
## [1] 0
\end{verbatim}

\begin{Shaded}
\begin{Highlighting}[]
\NormalTok{game <-}\StringTok{ }\NormalTok{game[}\DecValTok{2}\OperatorTok{:}\DecValTok{101}\NormalTok{]}
\NormalTok{game[}\DecValTok{1}\NormalTok{]}
\end{Highlighting}
\end{Shaded}

\begin{verbatim}
## [1] 1.495898e-07
\end{verbatim}

\begin{Shaded}
\begin{Highlighting}[]
\NormalTok{df <-}\StringTok{ }\KeywordTok{do.call}\NormalTok{(rbind, }\KeywordTok{Map}\NormalTok{(data.frame, }\DataTypeTok{point_pr=}\NormalTok{pr, }\DataTypeTok{game_pr=}\NormalTok{game))}
\KeywordTok{ggplot}\NormalTok{(df, }\KeywordTok{aes}\NormalTok{(}\DataTypeTok{x=}\NormalTok{point_pr, }\DataTypeTok{y=}\NormalTok{game_pr)) }\OperatorTok{+}
\StringTok{  }\KeywordTok{geom_point}\NormalTok{()}\OperatorTok{+}\KeywordTok{xlab}\NormalTok{(}\StringTok{'Probability of Winning a Service Point'}\NormalTok{)}\OperatorTok{+}\KeywordTok{ylab}\NormalTok{(}\StringTok{'Probability of Winning a Game'}\NormalTok{)}
\end{Highlighting}
\end{Shaded}

\includegraphics{22-isaac_files/figure-latex/unnamed-chunk-7-1.pdf}

\hypertarget{wnba-scores-eda}{%
\subsubsection{WNBA Scores (EDA)}\label{wnba-scores-eda}}

Q: What is the difference in PPG for a winning team at home vs a winning
team away? A:

\begin{Shaded}
\begin{Highlighting}[]
\NormalTok{wnba=}\KeywordTok{read.csv}\NormalTok{(}\StringTok{'~/Google Drive/My Drive/Sports Analytics/SportsAnalyticsBook/data/WNBA_Games2019_Scores.csv'}\NormalTok{)}
\KeywordTok{head}\NormalTok{(wnba)}
\end{Highlighting}
\end{Shaded}

\begin{verbatim}
##   Game         HomeTeam           AwayTeam Winner PTSwin PTSlose
## 1    1    Atlanta Dream       Dallas Wings   Home     76      72
## 2    2 New York Liberty      Indiana Fever   Away     81      80
## 3    3  Connecticut Sun Washington Mystics   Home     84      69
## 4    4   Minnesota Lynx        Chicago Sky   Home     89      71
## 5    5    Seattle Storm    Phoenix Mercury   Home     77      68
## 6    6   Las Vegas Aces Los Angeles Sparks   Home     83      70
##       WinningTeam
## 1   Atlanta Dream
## 2   Indiana Fever
## 3 Connecticut Sun
## 4  Minnesota Lynx
## 5   Seattle Storm
## 6  Las Vegas Aces
\end{verbatim}

\begin{Shaded}
\begin{Highlighting}[]
\KeywordTok{group_by}\NormalTok{(wnba, Winner)}\OperatorTok
\StringTok{  }\KeywordTok{summarize}\NormalTok{(}\DataTypeTok{Count=}\KeywordTok{n}\NormalTok{())}\OperatorTok
\StringTok{  }\KeywordTok{mutate}\NormalTok{(}\DataTypeTok{Percent=}\NormalTok{Count}\OperatorTok{/}\KeywordTok{sum}\NormalTok{(Count))}
\end{Highlighting}
\end{Shaded}

\begin{verbatim}
## # A tibble: 2 x 3
##   Winner Count Percent
##   <fct>  <int>   <dbl>
## 1 Away      80   0.392
## 2 Home     124   0.608
\end{verbatim}

\begin{Shaded}
\begin{Highlighting}[]
\KeywordTok{group_by}\NormalTok{(wnba, Winner)}\OperatorTok
\StringTok{  }\KeywordTok{summarize}\NormalTok{(}\DataTypeTok{Average=}\KeywordTok{mean}\NormalTok{(PTSwin,}\DataTypeTok{na.rm=}\NormalTok{T),}\DataTypeTok{sd=}\KeywordTok{sd}\NormalTok{(PTSwin,}\DataTypeTok{na.rm=}\NormalTok{T))}
\end{Highlighting}
\end{Shaded}

\begin{verbatim}
## # A tibble: 2 x 3
##   Winner Average    sd
##   <fct>    <dbl> <dbl>
## 1 Away      83.8  9.20
## 2 Home      84.8 10.8
\end{verbatim}

\begin{Shaded}
\begin{Highlighting}[]
\FloatTok{84.822-83.787}
\end{Highlighting}
\end{Shaded}

\begin{verbatim}
## [1] 1.035
\end{verbatim}

A home team winner scores on average 1.035 PPG more than an away team
winner.

\hypertarget{nfl}{%
\subsubsection{NFL}\label{nfl}}

\begin{Shaded}
\begin{Highlighting}[]
\NormalTok{nfl=}\KeywordTok{read.csv}\NormalTok{(}\StringTok{'~/Google Drive/My Drive/Sports Analytics/SportsAnalyticsBook/data/nfl_pbp.csv'}\NormalTok{)}
\NormalTok{nfl2 <-}\StringTok{ }\KeywordTok{select}\NormalTok{(nfl, }\KeywordTok{c}\NormalTok{(}\StringTok{'Date'}\NormalTok{,}\StringTok{'GameID'}\NormalTok{,}\StringTok{'qtr'}\NormalTok{,}\StringTok{'down'}\NormalTok{,}\StringTok{'time'}\NormalTok{,}\StringTok{'yrdline100'}\NormalTok{,}\StringTok{'ydstogo'}\NormalTok{,}\StringTok{'Yards.Gained'}\NormalTok{,}\StringTok{'Touchdown'}\NormalTok{,}\StringTok{'PlayType'}\NormalTok{,}\StringTok{'FieldGoalResult'}\NormalTok{,}\StringTok{'FieldGoalDistance'}\NormalTok{,}\StringTok{'ScoreDiff'}\NormalTok{,}\StringTok{'Season'}\NormalTok{))}
\KeywordTok{head}\NormalTok{(nfl2)}
\end{Highlighting}
\end{Shaded}

\begin{verbatim}
##         Date     GameID qtr down  time yrdline100 ydstogo Yards.Gained
## 1 2009-09-10 2009091000   1   NA 15:00         30       0           39
## 2 2009-09-10 2009091000   1    1 14:53         58      10            5
## 3 2009-09-10 2009091000   1    2 14:16         53       5           -3
## 4 2009-09-10 2009091000   1    3 13:35         56       8            0
## 5 2009-09-10 2009091000   1    4 13:27         56       8            0
## 6 2009-09-10 2009091000   1    1 13:16         98      10            0
##   Touchdown PlayType FieldGoalResult FieldGoalDistance ScoreDiff Season
## 1         0  Kickoff            <NA>                NA         0   2009
## 2         0     Pass            <NA>                NA         0   2009
## 3         0      Run            <NA>                NA         0   2009
## 4         0     Pass            <NA>                NA         0   2009
## 5         0     Punt            <NA>                NA         0   2009
## 6         0      Run            <NA>                NA         0   2009
\end{verbatim}

\hypertarget{th-down-analysis-eda}{%
\paragraph{4th Down Analysis (EDA)}\label{th-down-analysis-eda}}

Q: Using NFL Play by Play data, what percentage of the time do coaches
choose to go for it on 4th down? And what percentage of 4th down
attempts are successful? A:

\begin{Shaded}
\begin{Highlighting}[]
\CommentTok{# add indicator column for successful first down attempt}
\NormalTok{nfl2 <-}\StringTok{ }\NormalTok{nfl2 }\OperatorTok
\StringTok{  }\KeywordTok{mutate}\NormalTok{(}\DataTypeTok{FirstDown =} \KeywordTok{case_when}\NormalTok{(}
\NormalTok{    ydstogo }\OperatorTok{<}\StringTok{ }\NormalTok{Yards.Gained }\OperatorTok{~}\StringTok{ }\DecValTok{1}\NormalTok{,}
\NormalTok{    ydstogo }\OperatorTok{>}\StringTok{ }\NormalTok{Yards.Gained }\OperatorTok{~}\StringTok{ }\DecValTok{0}
\NormalTok{    ))}
\CommentTok{# filter by only plays on 4th down}
\NormalTok{down4 =}\StringTok{ }\KeywordTok{filter}\NormalTok{(nfl2, nfl2[}\StringTok{'down'}\NormalTok{]}\OperatorTok{==}\DecValTok{4}\NormalTok{)}

\CommentTok{#see what play types are run on first down and remove the noise}
\KeywordTok{group_by}\NormalTok{(down4,PlayType) }\OperatorTok
\StringTok{  }\KeywordTok{summarize}\NormalTok{(}\DataTypeTok{Count=}\KeywordTok{n}\NormalTok{())}\OperatorTok
\StringTok{  }\KeywordTok{mutate}\NormalTok{(}\DataTypeTok{Percentage=}\NormalTok{Count}\OperatorTok{/}\KeywordTok{sum}\NormalTok{(Count))}
\end{Highlighting}
\end{Shaded}

\begin{verbatim}
## # A tibble: 8 x 3
##   PlayType   Count Percentage
##   <fct>      <int>      <dbl>
## 1 Field Goal  7265  0.226    
## 2 No Play     1433  0.0446   
## 3 Pass        2239  0.0698   
## 4 Punt       19551  0.609    
## 5 QB Kneel      22  0.000685 
## 6 Run         1424  0.0444   
## 7 Sack         164  0.00511  
## 8 Timeout        1  0.0000312
\end{verbatim}

\begin{Shaded}
\begin{Highlighting}[]
\NormalTok{down4 =}\StringTok{ }\KeywordTok{filter}\NormalTok{(down4, down4[}\StringTok{'PlayType'}\NormalTok{]}\OperatorTok{!=}\StringTok{'No Play'} \OperatorTok{||}\StringTok{ }\NormalTok{down4[}\StringTok{'PlayType'}\NormalTok{]}\OperatorTok{!=}\StringTok{ 'QB Kneel'} \OperatorTok{||}\StringTok{ }\NormalTok{down4[}\StringTok{'PlayType'}\NormalTok{]}\OperatorTok{!=}\StringTok{ 'Timeout'}\NormalTok{)}

\CommentTok{# add indicator column for going for it on 4th}
\NormalTok{down4 <-}\StringTok{ }\NormalTok{down4 }\OperatorTok\StringTok{ }
\StringTok{  }\KeywordTok{mutate}\NormalTok{(}\DataTypeTok{GoForIt =} \KeywordTok{case_when}\NormalTok{(}
\NormalTok{    PlayType }\OperatorTok{==}\StringTok{ 'Pass'} \OperatorTok{~}\StringTok{ }\DecValTok{1}\NormalTok{,}
\NormalTok{    PlayType }\OperatorTok{==}\StringTok{ 'Run'} \OperatorTok{~}\StringTok{ }\DecValTok{1}\NormalTok{,}
\NormalTok{    PlayType }\OperatorTok{==}\StringTok{ 'Sack'} \OperatorTok{~}\StringTok{ }\DecValTok{1}\NormalTok{,}
\NormalTok{    PlayType }\OperatorTok{==}\StringTok{ 'Field Goal'} \OperatorTok{~}\StringTok{ }\DecValTok{0}\NormalTok{, }
\NormalTok{    PlayType }\OperatorTok{==}\StringTok{ 'Punt'} \OperatorTok{~}\StringTok{ }\DecValTok{0}
\NormalTok{  ))}
\CommentTok{# get percentage of 4th downs are gone for}
\KeywordTok{group_by}\NormalTok{(down4,GoForIt) }\OperatorTok
\StringTok{  }\KeywordTok{summarize}\NormalTok{(}\DataTypeTok{Count=}\KeywordTok{n}\NormalTok{())}\OperatorTok
\StringTok{  }\KeywordTok{mutate}\NormalTok{(}\DataTypeTok{Percentage=}\NormalTok{Count}\OperatorTok{/}\KeywordTok{sum}\NormalTok{(Count))}
\end{Highlighting}
\end{Shaded}

\begin{verbatim}
## # A tibble: 3 x 3
##   GoForIt Count Percentage
##     <dbl> <int>      <dbl>
## 1       0 26816     0.835 
## 2       1  3827     0.119 
## 3      NA  1456     0.0454
\end{verbatim}

\begin{Shaded}
\begin{Highlighting}[]
\CommentTok{# get percentage of successful attempted 4th downs }
\NormalTok{down4 }\OperatorTok
\StringTok{  }\KeywordTok{filter}\NormalTok{(down4[}\StringTok{'GoForIt'}\NormalTok{]}\OperatorTok{==}\DecValTok{1}\NormalTok{) }\OperatorTok
\StringTok{  }\KeywordTok{group_by}\NormalTok{(FirstDown) }\OperatorTok
\StringTok{    }\KeywordTok{summarize}\NormalTok{(}\DataTypeTok{Count=}\KeywordTok{n}\NormalTok{())}\OperatorTok
\StringTok{    }\KeywordTok{mutate}\NormalTok{(}\DataTypeTok{Percentage=}\NormalTok{Count}\OperatorTok{/}\KeywordTok{sum}\NormalTok{(Count))  }
\end{Highlighting}
\end{Shaded}

\begin{verbatim}
## # A tibble: 3 x 3
##   FirstDown Count Percentage
##       <dbl> <int>      <dbl>
## 1         0  1971     0.515 
## 2         1  1553     0.406 
## 3        NA   303     0.0792
\end{verbatim}

11\% of 4th downs are gone for and 40\% of those are successful,
regardless of how many yards to go there are

\hypertarget{football-sample-space-probability}{%
\paragraph{Football Sample Space
(Probability)}\label{football-sample-space-probability}}

A sample space contains all possible outcomes. An american football game
can either end with a win (W), loss (L) or a tie (T) which means our
sample space is \(\Omega = \{W,L,T \}\) and an event, \(E\) would be one
of the possible outcomes. If a team wins the game, the event for that
game would be \(E=\{W \}\) or if we want the event of the 2021 CSU
football season, it would be
\(E=\{ L, L, W, L, W, W, L, L, L, L, L, L \}\).


\end{document}
